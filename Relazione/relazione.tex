\documentclass[10pt,a4paper,openany,notitlepage]{article}
\usepackage[latin1]{inputenc}
\usepackage[italian]{babel}
\usepackage{hyperref}
\hypersetup{
    colorlinks,
    citecolor=gray,
    filecolor=red,
    linkcolor=blue,
    urlcolor=blue
}
\usepackage{amsmath}
\usepackage{amsfonts}
\usepackage{amssymb}
\usepackage{listings}
\lstset{%
	commentstyle=\color{green},
	frame=single,
	keepspaces=true,
	keywordstyle=\color{blue},
	numbers=left,
	numberstyle=\tiny\color{gray},
	rulecolor=\color{black},
	columns=flexibile
}
\author{Davide Polonio, Alessandro Bari}
\title{Progetto Dasi di Dati 2015}
\begin{document}
\maketitle
\newpage
\tableofcontents
\newpage

\section{Il progetto}
	\subsubsection{Studio di fattibilit\`a}
	\subsubsection{Raccolta e analisi dei requisiti}
		\paragraph{Negiozio "Linea Casa Bari"}
			Si vuole realizzare un database di un negozio che vende
                        \textit{prodotti} (codice prodotto, nome prodotto,
                        descrizione, categoria, prezzo, quantit\`a, ordine) di
                        diverse \textit{categorie} (nome categoria, IVA,
                        responsabile).  Il responsabile \`e un
                        \textit{dipendente} (codice dipendente, nome, cognome,
                        data assunzione, competenze). I prodotti sono
                        \textit{ordinati} (numero ordine, data, quantit\`a) ad
                        un \textit{Fornitore} (nome, indirizzo, telefono, Email,
                        fax, ordine). \textit{Iscritti} sono speciali clienti
                        che hanno diritto a sconti (codice iscritto, e-mail,
                        telefono, nome, cognome) in base ad un suo
                        \textit{storico} (codice iscritto, categoria, numero
                        acquisti), gli \textit{sconti} sono calcolati (numero
                        acquisti, categoria, %sconto).

\end{document}
