\subsection{Raccolta e analisi dei requisiti}
\subsubsection{Abstract}
\paragraph*{Negozio ''Linea Casa Bari''}
Si vuole realizzare una base di dati per un negozio al dettaglio che vende oggetti per la casa, di cui vogliamo rappresentare i dati dei clienti abituali, dei dipendenti, dei fornitori e dei prodotti, con le rispettive categorie. \\

Per i clienti \textbf{iscritti} (sono circa 20 e sono un sottoinsieme dei clienti totali), identificati da un codice, vogliamo tenere conto dell'identit\`a (nome,cognome), dell'indirizzo(via, numero, citt\`a, provincia) e dei contatti (e-mail,telefono). Ogni cliente iscritto ha un unico storico acquisti (insieme di scontrini presenti e passati) che specifica il numero di acquisti fatti su una certa categoria e da cui si derivano gli eventuali sconti di cui ha diritto. \\

Per ogni acquisto viene rilasciato uno \textbf{scontrino}, identificato da un codice scontrino e dalla data, deve essere riportata la quantit\`a relativa ad ogni prodotto (acquistato) e aggiornato il campo totale che tiene conto della spesa totale relativa ad un acquisto. \\

Per le 5 \textbf{categorie} di prodotti, identificate da un nome univoco (porcellane, pentolame, liste nozze, stoffe, stoviglie), vogliamo tenere conto della percentuale IVA a cui sono soggette. \\

Ad ogni categoria sono associati possibili \textbf{sconti} in base ad una distribuzione per livelli e con una relativa percentuale. \\

I \textbf{dipendenti}, responsabili ognuno di una singola categoria, sono identificati da un codice, inoltre vogliamo tenere conto dell'indirizzo (via,numero, citt\`a, provincia), dei recapiti (telefono, mail) e delle informazioni di base (nome, cognome, data di nascita, cod. fiscale). Ad ogni dipendente viene assegnato un determinato turno (mattina, pomeriggio) in un certo giorno lavorativo in cui sar\`a responsabile di una certa categoria. \\

Ogni \textbf{prodotto} appartiene ad una singola categoria, \`e identificato da un codice ed inoltre vogliamo tenere conto delle informazioni base (nome, descrizione, prezzo) e della quantit\`a disponibile (in magazzino e in esposizione). \\

Ogni prodotto disponibile deve essere stato precedentemente ordinato da un fornitore, con relativa \textbf{fattura}. La singola fattura deve essere identificata da un codice fattura univoco, inoltre teniamo conto della data e della quantit\`a riferita al singolo prodotto ordinato. \\

Per i \textbf{fornitori} (artigiani o grossisti), identificati da un nome univoco, dobbiamo tenere conto dei contatti (fax, mail, telefono), dell'indirizzo (via, numero, città, provincia).


\subsubsection{Glossario dei termini}
\begin{center}
\begin{tabular}{| l | p{5cm} | p{3cm} | p{3cm} |}
\hline

Termine & Descrizione & Sinonimi & Collegamenti \\ \hline

Iscritto & Compratore abituale iscritto a questa lista per avere diritto a sconti speciali. Pu\`o essere un dipendente. & Cliente abituale & Acquisto, Scontrino \\ \hline

Scontrino & Scontrino attestante lo storico degli acquisti. & Storico, acquisto & Iscritto, Prodotto \\ \hline

Categoria & 5 insiemi di prodotti. Un prodotto pu\`o appartenere ad un'unica categoria. Ogni categoria ha il suo univoco responsabile. & & Sconto, Dipende, Prodotto. \\ \hline

Sconto & Spetta solamente al cliente iscritto. Per ogni categoria esistono diversi livelli in base agli acquisti, a cui corrisponde una percentuale di sconto. & & Categoria. \\ \hline

Dipendente & Responsabile di una singola categoria. Pu\`o essere un cliente ma non un fornitore. & Responsabile & Categoria \\ \hline

Prodotto & Ogni prodotto pu\`o appartenere ad una sola categoria. & Oggetti, Prodotto ordinati o acquistati & Categoria, Fattura, Scontrino \\ \hline

Fattura & Pi\`u unit\`a di prodotto possono essere ordinate a fornitori diversi. Modifica il campo quantit\`a disponibile di prodotto. & Ordine & Fornitore, Prodotto \\ \hline

Fornitore & Forniscono i prodotti attraverso gli ordini. Un fornitore non pu\`o essere un cliente e pu\`o fornire prodotti di diverse categorie & & Ordine \\


\hline
\end{tabular}

\end{center}
