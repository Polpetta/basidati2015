\subsection{Raccolta e analisi dei requisiti}
\subsubsection{Abstract}
\paragraph*{Negozio ''Linea Casa Bari''}
Si vuole realizzare una base di dati per un negozio al dettaglio che vende oggetti per la casa, di cui vogliamo rappresentare i dati dei clienti abituali, dei dipendenti, dei fornitori e dei prodotti, con le rispettive categorie. \\

Per i clienti \textbf{iscritti} identificati da un codice, vogliamo tenere conto dell'identit\`a e dei suoi acquisti effettuati. \\

Ogni \textbf{prodotto} \`e identificato da un codice, vogliamo tenere conto delle informazioni base e della quantit\`a disponibile e catalogarlo in una delle cinque \textbf{categorie}, identificate come: porcellane, pentolame, liste nozze, stoffe, stoviglie. \newline

Ad ogni categoria sono associati possibili \textbf{sconti} in base ad una distribuzione per livelli e con una relativa percentuale. \\

I \textbf{dipendenti}, responsabili ognuno di una singola categoria, sono identificati da un codice e sono organizzati per turni. \\

Vogliamo tener conto degli ordini di merce ai \textbf{fornitori} con rispettive \textbf{fatture}. \\


\subsubsection{Glossario dei termini}
\begin{center}
\begin{tabular}{| l | p{5cm} | p{3cm} | p{3cm} |}
\hline

Termine & Descrizione & Sinonimi & Collegamenti \\ \hline

Iscritto & Compratore abituale iscritto a questa lista per avere diritto a sconti speciali. Pu\`o essere un dipendente. & Cliente abituale & Acquisto, Scontrino \\ \hline

Scontrino & Scontrino attestante lo storico degli acquisti. & Storico, acquisto & Iscritto, Prodotto \\ \hline

Categoria & 5 insiemi di prodotti. Un prodotto pu\`o appartenere ad un'unica categoria. Ogni categoria ha il suo univoco responsabile. & & Sconto, Dipende, Prodotto. \\ \hline

Sconto & Spetta solamente al cliente iscritto. Per ogni categoria esistono diversi livelli in base agli acquisti, a cui corrisponde una percentuale di sconto. & & Categoria. \\ \hline

Dipendente & Responsabile di una singola categoria. Pu\`o essere un cliente ma non un fornitore. & Responsabile & Categoria \\ \hline

Prodotto & Ogni prodotto pu\`o appartenere ad una sola categoria. & Oggetti, Prodotto ordinati o acquistati & Categoria, Fattura, Scontrino \\ \hline

Fattura & Pi\`u unit\`a di prodotto possono essere ordinate a fornitori diversi. Modifica il campo quantit\`a disponibile di prodotto. & Ordine & Fornitore, Prodotto \\ \hline

Fornitore & Forniscono i prodotti attraverso gli ordini. Un fornitore non pu\`o essere un cliente e pu\`o fornire prodotti di diverse categorie & & Ordine \\


\hline
\end{tabular}

\end{center}
