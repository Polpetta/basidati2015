\newpage
\subsubsection{Query}

\paragraph*{Query 1}
Prodotto pi\`u venduto (nota per visualizzare il risultato di questa query abbiamo omesso il campo descrizione)

\begin{verbatim}

+-------------+------------+--------+----------------+------------+-------------+
| CodProdotto | Nome       | Costo  | PercentualeIVA | Categoria  | Num_venduti |
+-------------+------------+--------+----------------+------------+-------------+
|           5 | padella    |  40.00 |             22 | Pentolame  |           5 |
|           6 | padella    |  34.50 |             22 | Pentolame  |           1 |
|          11 | Set. Tazze | 250.00 |             22 | Porcellane |           1 |
|          12 | tovaglia   |  10.00 |             22 | Tovaglie   |           2 |
|          14 | confetti   |   0.10 |             10 | ListeNozze |           1 |
|          15 | teglia     |  39.00 |             22 | Pentolame  |           1 |
|          16 | pentola    |  30.00 |             22 | Pentolame  |           2 |
+-------------+------------+--------+----------------+------------+-------------+

\end{verbatim}

\lstinputlisting[language=SQL,firstline=3,lastline=20,caption=Query 1]{res/code/query.sql}


\paragraph*{Query 2}
Prodotti invenduti

\begin{verbatim}

+-------------+
| CodProdotto |
+-------------+
|           1 |
|           2 |
|           3 |
|           4 |
|           7 |
|           8 |
|           9 |
|          10 |
|          13 |
+-------------+

\end{verbatim}

\lstinputlisting[language=SQL,firstline=25,lastline=28,caption=Query 2]{res/code/query.sql}

\paragraph*{Query 3}
Prodotti acquistati da un certo utente iscritto (per una migliore visualizzazione del risultato abbiamo omesso il campo Descrizione)

\begin{verbatim}

+-------------+------------+--------+----------------+------------+
| CodProdotto | Nome       | Costo  | PercentualeIVA | Categoria  |
+-------------+------------+--------+----------------+------------+
|           5 | padella    |  40.00 |             22 | Pentolame  |
|           5 | padella    |  40.00 |             22 | Pentolame  |
|           5 | padella    |  40.00 |             22 | Pentolame  |
|           5 | padella    |  40.00 |             22 | Pentolame  |
|           5 | padella    |  40.00 |             22 | Pentolame  |
|           6 | padella    |  34.50 |             22 | Pentolame  |
|          11 | Set. Tazze | 250.00 |             22 | Porcellane |
|          12 | tovaglia   |  10.00 |             22 | Tovaglie   |
|          14 | confetti   |   0.10 |             10 | ListeNozze |
|          16 | pentola    |  30.00 |             22 | Pentolame  |
|          15 | teglia     |  39.00 |             22 | Pentolame  |
|          16 | pentola    |  30.00 |             22 | Pentolame  |
+-------------+------------+--------+----------------+------------+

\end{verbatim}

\lstinputlisting[language=SQL,firstline=38,lastline=41,caption=Query 3]{res/code/query.sql}

\paragraph*{Query 4}
Fornitore da cui ho comprato di pi\`u (per una migliore visualizzazione del risultato abbiamo omesso il campo Fax e Telefono)

\begin{verbatim}

+----------+--------------------------+--------------------------+------------------------+
| Nome     |Mail                      | Indirizzo                | Numero_acquisto        |
+----------+--------------------------+--------------------------+------------------------+
| Arzenton |arzenton.scatole@gmail.com| via roma, 51 verona Vr   | 3                      |
+----------+--------------------------+--------------------------+------------------------+

\end{verbatim}

\lstinputlisting[language=SQL,firstline=45,lastline=55,caption=Query 4]{res/code/query.sql}

\paragraph*{Query 5}
Categoria che ha venduto pi\`u prodotti

\begin{verbatim}

+-----------+--------------+
| Categoria | Guadagno_Max |
+-----------+--------------+
| Pentolame |       511.00 |
+-----------+--------------+

\end{verbatim}

\lstinputlisting[language=SQL,firstline=59,lastline=70,caption=Query 5]{res/code/query.sql}

\paragraph*{Query 6}
Iscritti che non hanno mai comprato prodotti da una certa categoria (per una migliore visualizzazione del risultato abbiamo omesso il campo Fax e Telefono)

\begin{verbatim}

+-------------+------------+-----------+---------------------------+---------------------------+
| CodIscritto | Nome       | Cognome   |Mail                       | Indirizzo                 |
+-------------+------------+-----------+---------------------------+---------------------------+
|           1 | Alessandro | Bari      |alessandro.bari@gmail.com  | Via roma, 10 Verona Vr    |
|           2 | Carlo      | Sindaco   |carlo.sin@gmail.com        | Via S.paolo, 12 Verona Vr |
|           3 | Giovanni   | Mucciacia |giovanni.mu@gmail.com      | Via augusto, 12 Verona Vr |
+-------------+------------+-----------+------------+--------------+---------------------------+

\end{verbatim}

\lstinputlisting[language=SQL,firstline=75,lastline=82,caption=Query 6]{res/code/query.sql}

\paragraph*{Query 7}
Giorno della settimana dove vi \`e stato il maggior guadagno

\begin{verbatim}

+------------+-------------+
| Data       | Tot_vendite |
+------------+-------------+
| 2015-01-06 |      381.00 |
+------------+-------------+

\end{verbatim}

\lstinputlisting[language=SQL,firstline=86,lastline=90,caption=Query 7]{res/code/query.sql}

\paragraph*{Query 8}
Nome dipendente responsabile della categoria che ha stampato il maggior numero di scontrini

\begin{verbatim}

+----------+---------------+---------------+
| Nome     | Num_Scontrini | NomeCategoria |
+----------+---------------+---------------+
| Giovanni |             9 | Pentolame     |
+----------+---------------+---------------+

\end{verbatim}

\lstinputlisting[language=SQL,firstline=94,lastline=108,caption=Query 8]{res/code/query.sql}

\paragraph*{Query 9}
Per ogni iscritto il livello di sconto piu alto nella categoria che ha venduto di pi\`u questo mese

\begin{verbatim}

+-------------+------------+---------+-----------------+
| CodIscritto | Nome       | Cognome | Livello_massimo |
+-------------+------------+---------+-----------------+
|           1 | Alessandro | Bari    |               1 |
+-------------+------------+---------+-----------------+

\end{verbatim}

\lstinputlisting[language=SQL,firstline=112,lastline=140,caption=Query 9]{res/code/query.sql}

\paragraph*{Query 10}
Per iscritto i suoi scontrini (numero di acquisti) e i livelli (attualmente raggiunti) per ogni categoria.
Nell'esempio visualizziamo gli scontrino dell'utente $1$.

\begin{verbatim}

+-----------------+---------------+-----------------+
| Numero_Acquisti | NomeCategoria | Livello_attuale |
+-----------------+---------------+-----------------+
|               2 | ListeNozze    |               1 |
|              18 | Pentolame     |               1 |
|               2 | Porcellane    |               1 |
|               2 | Tovaglie      |               1 |
+-----------------+---------------+-----------------+

\end{verbatim}

\lstinputlisting[language=SQL,firstline=144,lastline=161,caption=Query 10]{res/code/query.sql}
