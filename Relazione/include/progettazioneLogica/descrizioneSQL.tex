\subsubsection{Create Table}
Prima di creare le tabelle abbiamo inserito le seguenti righe di codice:

\lstinputlisting[language=SQL,firstline=1,lastline=11]{res/code/createtable.sql}
In modo da non incorrere in errori in caso di ricreazione delle tabelle. \newline
I create table veri e propri sono i seguenti:

\lstinputlisting[language=SQL,firstline=13,lastline=15]{res/code/createtable.sql}
crea la tabella Categoria con chiave primaria NomeCategoria;

\lstinputlisting[language=SQL,firstline=16,lastline=27]{res/code/createtable.sql}
crea la tabella Sconto che ha come chiave primaria l'attributo Id;

\lstinputlisting[language=SQL,firstline=28,lastline=36]{res/code/createtable.sql}
Crea la tabella Scaglione (relazione tra Sconto e Categoria), ha come chiave entrambi i suoi campi (Categoria e Sconto) che sono chiave esterna per Categoria e Sconto rispettivamente. Sia Categoria che Sconto hanno il vincolo DELETE ON CASCADE in quanto vogliamo che alla cancellazione di una Categoria anche gli sconti a questa associata vengano cancellati (con l'intervento aggiuntivo del trigger delete\_categoria spiegato successivamente), mentre per l'eliminazione di Sconto deve essere eliminata la tupla corrispondente di Scaglione ma mantenuta la Categoria associata (in quanto potrebbe essere non vuota);

\lstinputlisting[language=SQL,firstline=38,lastline=54]{res/code/createtable.sql}
Crea tabella Dipendente con chiave primaria CodDipendente e Categoria come chiave esterna con vincolo ON DELETE CASCADE in modo tale che all'eliminazione (evento raro ma possibile) di una categoria anche il dipendente venga eliminato;

\lstinputlisting[language=SQL,firstline=56,lastline=67]{res/code/createtable.sql}
crea la tabella Prodotto con chiave primaria CodProdotto e chiave esterna Categoria. In caso di cancellazione di categoria agisce il trigger delete\_categoria che impedisce l'eliminazione in caso di esistenza di prodotti in quella categoria;

\lstinputlisting[language=SQL,firstline=68,lastline=80]{res/code/createtable.sql}
crea la tabella Iscritto con chiave primaria CodIscritto;

\lstinputlisting[language=SQL,firstline=82,lastline=93]{res/code/createtable.sql}
crea la tabella Scontrino con chiave primaria Id e chiave esterna Iscritto a cui viene associato il vincolo ON DELETE CASCADE che alla cancellazione di un iscritto procede alla cancellazione di tutti i suoi scontrini (qui interviene il trigger delete certifica che verr\`a spiegato successivamente). Scontrino si riferisce alle singole righe di uno scontrino, lo scontrino totale viene identificato dal CodScontrino. Abbiamo fatto questa scelta in quanto la maggior parte delle azioni sul database vengono eseguite sulle singole righe e non sullo scontrino totale (questo fatto si ritrova anche nella tabella Fattura);

\lstinputlisting[language=SQL,firstline=95,lastline=102]{res/code/createtable.sql}
crea la tabella Certifica (relazione tra prodotto e scontrino) con chiave primaria Prodotto e Scontrino entrambe anche chiavi esterne per le tabelle Prodotto e Scontrino rispettivamente;

\lstinputlisting[language=SQL,firstline=104,lastline=112]{res/code/createtable.sql}
crea la tabella Fornitore con chiave primaria Nome in quanto univoco;

\lstinputlisting[language=SQL,firstline=114,lastline=123]{res/code/createtable.sql}
crea la tabella Fattura, che si riferisce alle singole righe di fattura, la fattura totale viene identificata dal CodFattura. Abbiamo fatto questa scelta in quanto la maggior parte delle azioni sul database vengono eseguite sulle singole righe e non sulla fattura totale. La chiave primaria \`e Id;

\lstinputlisting[language=SQL,firstline=125,lastline=133]{res/code/createtable.sql}
crea la tabella Registrato (relazione tra Prodotto e Fattura) con chiave primaria Prodotto e Fattura entrambe chiavi esterne per le relazioni Prodotto e Fattura rispettivamente;


\subsubsection{Procedure}

\lstinputlisting[language=SQL,caption=Nuovo Livello]{res/code/Procedura_NuovoLivello.sql}

Questa Procedure esegue diversi controlli:
\begin{enumerate}

\item Controlla la percentuale di sconto inserita sia maggiore di zero

\item Se la categoria collegata al livello di sconto esiste

\item Se non esiste gi\`a lo stesso livello che deve essere aggiunto

\item Controlli sulla scalarit\`a per stessa categoria di riferimento. In particolare controlla se non esistono altri livelli con numero livello pi\`u alto, percentuale sconto pi\`u alto o tetto massimo pi\`u alto; in quanto non avrebbe senso l'inserimento altrimento.
\end{enumerate}

Se queste condizioni sono negate viene generato un errore, altrimenti procede all'inserimento di un nuovo livello di sconto relativo ad una categoria ricevuta come input (la categoria deve esistere a priori).
