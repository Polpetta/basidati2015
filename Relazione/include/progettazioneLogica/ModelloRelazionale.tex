\subsubsection{Modello Relazionale}

Seguendo i procedimenti di trasformazione dello schema-ER al modello Relazionale abbiamo ottenuto: \newline
PRODOTTO (\underline{CodProdotto}, Nome, Descrizione, Quantit\`a, Costo, PercentualeIVA, Categoria) \newline
SCONTRINO (\underline{Prodotto}, \underline{Data}, \underline{CodScontrino}, Quantit\`a, Subtotale, Iscritto) \newline
CERTIFICA (\underline{Prodotto}, \underline{Data}, \underline{CodScontrino}) \newline
FATTURA (\underline{Prodotto}, \underline{CodFattura}, Data, Quantit\`a, Fornitore) \newline
REGISTRATO (\underline{Prodotto}, \underline{Codice Fattura}) \newline
CATEGORIA (\underline{Nome Categoria}) \newline
SCONTO (\underline{Categoria}, \underline{Livello}, \underline{PercentualeSconto}, \underline{TettoMax}) \newline
SCAGLIONI (\underline{Categoria}, \underline{Livello}) \newline
ISCRITTO (\underline{CodIscritto}, Nome, Cognome, Fax, Telefono, Mail, Indirizzo) \newline
FORNITORE (\underline{Nome}, Fax, Telefono, Mail, Indirizzo) \newline
DIPENDENTE (\underline{CodDipendente}, Nome, Cognome, Data Nascita, Codice Fiscale, Telefono, Mail, Data Inizio, Categoria) \\

Qui vengono espressi i vincoli non esprimibili nel modello relazionale:
\begin{itemize}
\item In PRODOTTO vincolo d'integrit\`a di categoria con CATEGORIA
\item In SCONTRINO vincolo d'integrit\`a di prodotto con CERTIFICA
\item In SCONTRINO vincolo d'integrit\`a di iscritto con ISCRITTO
\item In CERTIFICA vincolo d'integrit\`a di prodotto con PRODOTTO
\item In CERTIFICA vincolo d'integrit\`a di codice scontrino con SCONTRINO
\item In CERTIFICA vincolo d'integrit\`a di data con SCONTRINO
\item In FATTURA vincolo d'integrit\`a di prodotto con REGISTRATO
\item In FATTURA vincolo d'integrit\`a di fornitore con FORNITORE
\item In REGISTRATO vincolo d'integrit\`a di prodotto con PRODOTTO
\item In REGISTRATO vincolo d'integrit\`a di codice fattura con FATTURA
\item In SCONTO vincolo d'integrit\`a di categoria con SCAGLIONI
\item In SCAGLIONI vincolo d'integrit\`a di categoria con CATEGORIA
\item In SCAGLIONI vincolo d'integrit\`a di livelli con SCONTO
\item In DIPENDENTE vincolo d'integrit\`a di categoria con CATEGORIA \\

\end{itemize}

Inoltre per alcuni attributi abbiamo attribuito le seguenti propriet\`a:
\begin{itemize}
\item In PRODOTTO Descrizione, Foto possono essere NULL
\item In ISCRITTO Fax pu\`o essere NULL
\item In FORNITORE Fax pu\`o essere NULL
\item In DIPENDENTE Mail pu\`o essere NULL
\item In SCONTO il tetto-max pu\`o essere NULL

\end{itemize}

