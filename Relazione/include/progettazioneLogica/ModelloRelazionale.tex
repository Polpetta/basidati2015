\subsubsection{Modello Relazionale}

Seguendo i procedimenti di trasformazione dello schema-ER al modello Relazionale abbiamo ottenuto: \newline
PRODOTTO (\underline{CodProdotto}, Nome, Descrizione, Quantit\`a, Costo, PercentualeIVA, Categoria) \newline
SCONTRINO (\underline{Id}, Data, CodScontrino, Quantit\`a, Subtotale, Iscritto) \newline
CERTIFICA (\underline{Prodotto}, \underline{Scontrino}) \newline
FATTURA (\underline{Id}, CodFattura, Data, Quantit\`a, Fornitore) \newline
REGISTRATO (\underline{Prodotto}, \underline{Fattura}) \newline
CATEGORIA (\underline{Nome Categoria}) \newline
SCONTO (\underline{Id}, Livello, PercentualeSconto, TettoMax) \newline
SCAGLIONI (\underline{Categoria}, \underline{Sconto}) \newline
ISCRITTO (\underline{CodIscritto}, Nome, Cognome, Fax, Telefono, Mail, Indirizzo, Password) \newline
FORNITORE (\underline{Nome}, Fax, Telefono, Mail, Indirizzo) \newline
DIPENDENTE (\underline{CodDipendente}, Nome, Cognome, Data Nascita, Codice Fiscale, Telefono, Mail, Data Inizio, Indirizzo, Categoria, Password) \\

Inoltre per alcuni attributi abbiamo attribuito le seguenti propriet\`a:
\begin{itemize}
\item In PRODOTTO Descrizione, Foto possono essere NULL
\item In ISCRITTO Fax pu\`o essere NULL
\item In FORNITORE Fax pu\`o essere NULL
\item In DIPENDENTE Mail pu\`o essere NULL
\item In SCONTO il TettoMax pu\`o essere NULL

\end{itemize}

